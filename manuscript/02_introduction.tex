\chapter*{Introduction}


Coronavirus disease 2019 (COVID-19), caused by the novel human pathogen severe acute respiratory syndrome coronavirus 2 (SARS-CoV-2), is a highly transmissible disease that has resulted in a widespread global pandemic \citep{hu}. The understanding of the immunology of COVID-19 has rapidly evolved since early 2020, with a focus on vaccine development. From December 2020 to June 2021, 7 different vaccines have been listed for World Health Organization (WHO) Emergency Use Listing. As of 30 August 2021, a total of 5,019,907,027 vaccine doses have been administered worldwide \citep{who}.

The adaptive immune system is key for a successful response to most viral infections. It is composed by three main elements: B cells, which produce antibodies, CD4\textsuperscript{+} T cells with helper and effector functionalities and CD8\textsuperscript{+} T cells that kill infected cells. The activaction of these cells relies on the recognition of foreign antigenic proteins. Neutralizing antibodies bind to regions of viral antigens (called epitopes) located in the protein surface and aim to block the attachment of the virus to the human host cell, thus preventing cell infection. Most current vaccines aim to produce an antibody response, but although it is critical for virus neutralization and disease control, B cell responses to \covid{} have limited duration and breadth \citep{vaccinetcell}. The role of T cells in COVID-19 infection and their importance in vaccines is gaining interest among the scientific community since T cells are major mediators of long-term memory and persist much longer than antibodies \citep{tcellsdiag}. The importance of T cells is further supported by the T cell lymphopenia (low lymphocyte counts in peripheral blood) upon COVID-19 infection that correlates with disease severity \citep{lymphopeniaseverity}.

% The degree of lymphopenia correlates with disease severity, and it is usually associated with a pro-inflamatory cytokine storm \citep{lymphopeniaseverity}. It is hypothesized that the underlying causes of lymphopenia can be the pro-inflamatory cytokine levels, exhaustion of T cells upon COVID-19 infection and direct \covid-infection of T cells \citep{lymphopenia}.

The T cell receptors (TCR), located on the cellular membrane surface, are the T cell equivalent of B cell receptors (a membrane-bound version of antibodies). Unlike antibodies, these receptors are not capable of direct binding to a viral protein, but they require that it has been previously processed either by infected cells or by antigen presenting cells. These cells then display the antigenic epitopes on their major histocompatibility complex (MHC) surface membrane molecules, and the TCR binds to both the MHC and the epitope before its activation.

\textit{TO-DO:
(MHC I CD8, MHC II CD4, tcr specificity, $10^15$, tcr generation)
(Covid: proteome, epitopes)
(Analysis of TCR repertoires: focus on network analysis and antigen specificity annotation, MIRA assays)
(vaccines, first tcr studies)
(Objectives)}


% \textit{Are T cell repertoires useful as diagnostics for SARS- CoV- 2 infection?
% CD4 and CD 8 T cells are found in patients
% and recognize epitopes from most viral proteins. This was
% predicated to some extent by what was seen when examining
% T cell responses to infection with the original SARS-CoV caus­
% ing the 2002/2003 SARS epidemic. Here, specific T cells were
% recognized for their ability to respond to peptide pools
% derived from viral proteins [9]. T cells recognized mostly N,
% M, and S proteins. CD8 T cell responses were more common
% than CD4 T cell responses. There was also profound lympho­
% depletion in patients with severe disease. This is almost exactly
% what is now seen in SARS-CoV-2 infections.CD4 and CD 8 T cells are found in patients
% and recognize epitopes from most viral proteins. This was
% predicated to some extent by what was seen when examining
% T cell responses to infection with the original SARS-CoV caus­
% ing the 2002/2003 SARS epidemic. Here, specific T cells were
% recognized for their ability to respond to peptide pools
% derived from viral proteins [9]. T cells recognized mostly N,
% M, and S proteins. CD8 T cell responses were more common
% than CD4 T cell responses. There was also profound lympho­
% depletion in patients with severe disease. This is almost exactly
% what is now seen in SARS-CoV-2 infections.}





% \textcolor{lightgray}{\lipsum[1-15]}
