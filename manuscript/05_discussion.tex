\chapter*{Discussion}
\addcontentsline{toc}{chapter}{Discussion}\label{cap:dis}


The current study presents an in-depth characterization of TCR repertoires after COVID-19 disease and vaccination. Analyzing these repertoires from the frequency space revealed low proportions of T cells and less TCR diversity (more clonality) in convalescent individuals compared to healthy vaccine and placebo recipients, which was consistent with the clinical characteristics of the COVID-19 disease: lymphopenia and antigen-specific clonal expansions that reduce the repertoire size and diversity.

Adding 11 convalescent individual samples to match the sample size of the vaccine recipients cohort increased the statistical power of the comparisons between both groups, and both convalescent cohorts TCR repertoires beign amplified by the same assay minimized potential batch effects. This strategy could not be applied to placebo, since there are no further published data, but an alternative would be to add a cohort of unexposed healthy controls, on which there are many TCR repertoires published in immuneACCESS database.

Response to \covid{} was measured in terms of breadth and depth, where immunization to the spike protein in convalescent and vaccinated cohorts and a high response to non-spike antigens exclusive to convalescent subjects were observed, acknowledging that infection provides immunization against several virus proteins, while vaccines only elicit immunization against the single antigen they carry (the spike protein). The results were consistent with those shown in the data source publication \citep{janssen}, even though a different dataset of \covid-specific TCRs was used to annotate the bulk TCR repertoires. In \cite{janssen}, authors claim to have used a TCR dataset determined to be \covid-specific and enriched in subjects with natural infection relative to placebos curated by \cite{snyder} from public MIRA datasets \citep{immunecode}. However, this study is still a preprint and the data have not yet been made publicly available.

The workaround was to use an enriched \covid-specific TCR dataset generated by \cite{metaclonotypes} from the same MIRA data in the immuneCODE database, but with a different method than \cite{snyder}. Because CD8\textsuperscript{+} MIRA dataset was made publicly available before the CD4\textsuperscript{+} dataset, the authors have only curated and published the enriched dataset of \covid-specific CD8\textsuperscript{+} TCRs to this date. However, as both the input data (MIRA unenriched CD4\textsuperscript{+} TCRs) and the code were public when the present work was being made, the enrichment analysis could be performed to extract 64 new \covid-specific enriched TCRs.

In-depth analysis of the location of the epitopes recognized by TCRs showed that, in COVID-19 convalescent individuals, epitopes of S and N proteins are immunodominant amongst coronavirus antigens, as in the case of antibodies immune response \citep{abepitopes}. It was also observed that in some cases, in natural infection non-spike responses are stronger, particularly to epitopes of N protein, as it has been previously reported \citep{targetstcells, snyder}. The response to multiple epitopes of the same antigen may provide long-lasting protection to \covid, specially with the emergence of new variants. Response to spike protein of convalescent and vaccine recipient individuals was undistinguishable in terms of breadth and depth, but also 4 out of 5 spike epitopes were recognized by TCRs in both cohorts, remarking the similarity of infection and vaccine immunization. The only epitope exclusive to vaccine recipients (S4) was the most buried in the protein 3D structure, raising the question of whether antigenic processing of the whole virus versus the isolated S protein can generate different epitopes.

Studying TCR repertoires only from the frequency space (in terms of clonal expansions and diveristy) or from the sequence space simply looking for the presence or absence of certain sequences ignores the huge inter-individual TCR variability and may reveal only a tiny fraction of the true underlying signal of an immune response in the repertoire. To get a fuller and unexplored picture of TCR response in COVID-19 disease and vaccination, TCR repertoires were studied from a sequence similarity perspective. Given the vast diversity of sequences in a TCR repertoire, pairwise distances calculation can become a computationally expensive task. Currently, there are several existing tools for TCR sequence similarity analysis \citep{gliph2, alice, metaclonotypes}. tcrdist3, although it is one of the most time-consuming algorithms, was chosen among all the alternatives for having a biochemically informed distance metric that accounts for sequence similarities from an aminoacid properties perspective.

This allowed to build networks of TCR sequence similarity that revealed that spike-specific TCRs in convalescent and vaccinated cohorts acted as network hubs, beign connected to several similar sequences and often to itself, as the TCR generation process can converge and generate exact same CDR3 aminoacid sequences with different nucleotide sequences. Non-spike-specific TCRs were present in the three cohorts of this study, but the network analysis perspective showed that they had low hub scores in vaccinated and placebo repertoires, indicating that, as it would be expected, neither of them had experienced an immune response against \covid{} non-spike antigens, but may have cross-reactive TCRs. The presence of putatively cross-reactive TCRs in the \covid-specific dataset used for annotation denotes that the enrichment of MIRA datasets can be further fine-tuned. It has been found that the origin of these cross-reactive TCRs may be previous exposure to other common cold coronaviruses \citep{pfizertcr}.

Although the present work provides a deeper understanding of \covid{} immunization in disease and vaccination, one of its limitations relies on the repertoires antigen-specificity annotation process. Responses to \covid{} epitopes could be studied with a sequence similarity network approach if and only if at least one \covid-specific TCR was found in the repertoire via a V gene and CDR3 aminoacid sequence exact match. This ignores the presence in the repertoire of TCRs highly similar (but not exact matches) to those in the \covid-specific dataset. One way to improve this work would consist of screening the repertoires with a fuzzy match approach and establishing a tcrdist3 distance threshold, but note that this would also increase the chances of finding unspecific TCRs, that can be subsequently evaluated according to its node metrics in the network analysis.

Another limitation to further extend this analysis is the lack of vaccine recipients TCR repertoire data to track the immune response at the population level. Vaccine developers are gradually publishing repertoire data, but currently the cohorts are very small. As these datasets become available, characteristics of TCR responses to vaccines based on different technologies (viral vector, mRNA) could be compared. Also, the identification of novel \covid-specific TCRs could benefit from more TCR repertories of convalescent individuals bearing different MHC alleles combination, since the MHC alleles determine which epitopes are preferentially presented and thus which TCRs are going to be expanded in the repertoire.

All in all, the great effort of the scientific community in \covid-specific immune repertoire data generation has quickly made possible to track certain TCRs as indicators of past and ongoing \covid{} immunization, which may be detected in blood at least two months past exposure \citep{snyderdiagnosis}. Thanks to the boosted research in COVID-19 immunology, we are closer than ever to defining TCRs as biomarkers, and these advances set a precedent of best practices for studying TCR repertoires in other infections and diseases.



% In general, all TCR repertoire antigen-specificity studies are limited by biases such as not covering all the possible antigen epitopes in antigen-enrichment \textit{in vitro} assays and the low inter-individual overlap of TCR sequences due to MHC alleles: certain MHC alleles preferentially present certain epitopes and that uniquely shapes an individual's TCR repertoire.

%% MIRA limitations: certain epitopes, low overlap inter-individual, difficult, no hla

%% (Perspectives, single cell TCRrep)
