\chapter*{Abstract}
\addcontentsline{toc}{chapter}{Abstract / Resumen}\label{cap:abs}

\setcounter{page}{1}

Immunization against \covid{} following infection or vaccination has been extensively studied from the perspective of antibody responses. However, T cells also play an important role in protection against COVID-19, conferring in some cases a more durable immunization. Signals of past and present infections are encoded in the set of up to 10\textsuperscript{10} different T cell receptors (TCRs) that bind to antigens to trigger an immune response. A tremendous effort has been made for \textit{in vitro} identification of \covid-specific TCR sequences, and in 2021 the first TCR repertoire studies in disease and vaccination have been published. Nevertheless, these analyses are often shallow and centered only in \covid{} TCRs, disregarding the information from the rest of the repertoire. In the present study, published TCR repertoires of 19 COVID-19 convalescent, 19 Ad26.COV2.S (Janssen) vaccinated and 5 placebo recipient individuals have been re-analyzed with a different \covid-specific TCRs dataset and from a sequence similarity perspective. Convalescent and vaccinated cohorts exhibited a similar TCR response to the viral spike protein --the single antigen in the vaccine-- in terms of breadth, depth and epitope location, whereas response to non-spike antigens was significantly higher in convalescent subjects compared with vaccinated and placebo cohorts. Furthermore, sequence similarity network analysis revealed that putatively unspecific TCRs (i.e. any \covid{} TCRs in placebos and non-spike TCRs in vaccinated) are less connected in the repertoire network, suggesting that they may be cross-reactive to other antigens. These findings aim to broaden the understanding of how \covid{} exposure shapes the TCR repertoire in disease and vaccination.

\vspace{1cm}


\chapter*{Resumen}

La inmunización frente a \covid{} tras la infección o la vacunación ha sido ampliamente estudiada desde la perspectiva de las respuestas de anticuerpos. Sin embargo, las células T también tienen un papel importante en la protección contra la COVID-19, confiriendo en algunos casos una inmunidad más duradera. Las señales de infecciones pasadas y presentes quedan reflejadas en el conjunto de hasta 10\textsuperscript{10} receptores de células T (TCRs) distintos que se unen a antígenos para desencadenar una respuesta inmunitaria. Se han realizado grandes esfuerzos para la identificación \textit{in vitro} de secuencias de TCR específicas de \covid{}, y en 2021 se han publicado los primeros estudios de repertorios de TCR tras la enfermedad y la vacunación. No obstante, estos análisis suelen ser someros y se centran exclusivamente en los TCRs asociados a \covid{}, despreciando la información del resto del repertorio. En el presente estudio, repertorios de TCR públicos de 19 individuos convalecientes de COVID-19, 19 individuos que recibieron la vacuna Ad26.COV2.S (Janssen) y 5 individuos que recibieron un placebo han sido analizados de nuevo con un conjunto distinto de TCRs asociados a \covid{} y con un enfoque de similitud de secuencia. Los individuos convalecientes y vacunados exhibieron una respuesta similar de TCR ante la proteína \textit{spike} --el único antígeno presente en la vacuna-- en términos de amplitud, profundidad y localización de epítopos, mientras que la respuesta al resto de antígenos virales fue significativamente más alta en los sujetos convalecientes en comparación con las cohortes de vacunados y placebos. Además, el análisis de redes de similitud de secuencia reveló que aquellos TCRs putativamente inespecíficos (cualquier TCR asociado a \covid{} en placebos y TCRs asociados a proteínas distintas a \textit{spike} en vacunados) están menos conectados en la red del repertorio, sugiriendo que pueden presentar reactividad cruzada con otros antígenos. Estos hallazgos pretenden ampliar el conocimiento sobre cómo la exposición a \covid{} influye en la composición del repertorio de TCR tras la enfermedad y la vacunación.






\newpage

% TCR repertoires have low inter-individual overlap, which hinders biomarker discovery.
