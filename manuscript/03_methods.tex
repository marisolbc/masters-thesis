\chapter*{Material and Methods}
\label{chap:matmed}


\section*{TCR data}

This study is based on public data from three previous works \citep{janssen, immunecode, metaclonotypes}.

The dataset used for TCR repertoire analysis \citep{janssen} includes samples from 32 individuals: 8 convalescent from COVID-19, 19 who received the Ad26.COV2.S vaccine developed by Janssen Pharmaceutica during a clinical trial, and 5 subjects who received a placebo. Peripheral blood samples were collected post diagnosis or vaccination and immunosequencing of the CDR3 regions of human \TCRB{} chains was performed with the immunoSEQ Assay (Adaptive Biotechnologies). Data was accessed on July 2021 via Adaptive Biotechnologies immuneACCESS\textregistered{} database (immuneACCESS\textregistered{} DOI: \url{https://doi.org/10.21417/GA2021N}).

To match the sample size of vaccinated individuals with data generated with the same procedure, 11 TCR repertoire samples from COVID-19-convalescent subjects were randomly selected from the COVID-19-HUniv12Oct dataset on Adaptive Biotechnologies ImmuneCODE\texttrademark{} project \citep{immunecode}. The full dataset contains \TCRB{} repertoires from 193 convalescent patients whose blood sample was collected at the Hospital Univesitario 12 de Octubre (Madrid, Spain). Data was accessed on Aug 2021 via Adaptive Biotechnologies immuneACCESS\textregistered{} database (immuneACCESS\textregistered{} DOI: \url{https://doi.org/10.21417/ADPT2020COVID}, ImmuneCODE-COVID-Release-002).

% TO-DO: add samples IDs
%  further enrich this dataset in convalescent individuals

\covid-specific CD8\textsuperscript{+} \TCRB{} sequences were obtained from \cite{metaclonotypes}. This sequences are proven to bind \covid{} epitopes by Multiplex Identification of Receptor Antigen (MIRA) assays \citep{immunecode} and are also enriched in bulk \TCRB{} repertoires of convalescent individuals compared to healthy controls. For the present study, only \TCRB{} sequences with a strong evidence of HLA restriction (N = 1831) were taken into consideration.
%%% TO-DO: limit Holm 005 ???







\section*{\covid-specific CD4\textsuperscript{+} TCRs discovery}

While \covid-specific CD4\textsuperscript{+} have been used to annotate TCR repertoires in previous studies \citep{janssen, gittelman2021diagnosis}, those enriched and high-reliable datasets are not currently public. ImmuneCODE\texttrademark{} project contains an unenriched dataset of 6809 CD4\textsuperscript{+} TCRs that bind 49 different \covid{} epitopes presented by class II MHC molecules in MIRA assays. Data was accessed on Aug 2021 via Adaptive Biotechnologies immuneACCESS\textregistered{} database (immuneACCESS\textregistered{} DOI: \url{https://doi.org/10.21417/ADPT2020COVID}, ImmuneCODE-COVID-Release-002).

These TCRs were further screened for enrichment compared to a background of healthy individuals repertoires in order to remove TCRs that may be highly public or cross-reactive to common antigens. 64 TCRs were selected to annotate the repertoires, in addition to CD8\textsuperscript{+} dataset. The enrichment analysis was performed with tcrdist3 Python toolkit (Docker image v0.1.9) \citep{metaclonotypes, tcrdist}, following the same meta-clonotype discovery pipeline employed for \covid{} CD8\textsuperscript{+} TCR discovery as in \cite{metaclonotypes}.






\section*{Measurement of T-cell response to \covid}

The 43 \TCRB{} repertoires were annotated for antigen-specificity with the \covid-specific TCRs (CD4\textsuperscript{+} and CD8\textsuperscript{+}) by matching CDR3 aminoacid sequence and V gene. The \covid{} response  of each individual to spike and non-spike proteins was measured in terms of breadth, defined as the proportion of distinct TCRs recognizing certain protein among all the unique sequences in a repertoire, and in terms of depth, which is the proportion of the frequency of those \covid-specific TCRs.

%%% TO-DO: breadth and depth fornulas?





\section*{CDR3 pairwise distances}

Pairwise distances between all CDR3 sequences in a given sample were computed with tcrdist3 \citep{metaclonotypes, tcrdist}, which implements a custom distance metric based on BLOSUM62 substitution matrix to account for similar aminoacid substitutions, and applies different weights depending on the importance of every CDR3 position in antigen binding. Total runtime was \texttildelow 103 hours with parallel processing (40 CPUs, 256 GB of RAM).





\section*{Network analysis}

In this analysis each unique CDR3 aminoacid sequence were considered as a node. An edge was built between two nodes if their pairwise distance was $\leq$ 12. The reason behind this threshold is that 12 is the greatest possible distance between two CDR3 with one mismatch according to tcrdist3 algorithm. Networks were built and analyzed with R igraph package v1.2.6 \citep{igraph}.

\section*{Data analysis and visualization}

All plots and analyses were carried in R 3.6.1 \citep{r}. For data analysis, the packages dplyr v1.0.2 \citep{dplyr}, tidyr v1.1.2 \citep{tidyr}, rstatix v0.7.0 \citep{rstatix} and parallel v3.6.1 \citep{r} were used. Networks were plotted with the ggraph v2.0.2 package \citep{ggraph}. All other plots were generated with ggplot2 v3.3.2 \citep{ggplot2} and ggpubr v0.4.0 \citep{ggpubr}. 3D visualization of the SARS-CoV-2 spike protein (PDB ID: 6XR8) was generated with Protein Imager \citep{proteinimager}.
%%% TO-DO: github repo here?
